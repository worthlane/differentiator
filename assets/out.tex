\documentclass[12pt,a4paper]{extreport}
\input{style}
\title{<<I fucking love science>>}
\author{Maklakov Artyom B05-332}
\usepackage[ sorting = none, style = science ]{biblatex}
\addbibresource{library.bib}
\begin{document}
\maketitle
\tableofcontents
\section{Getting superhard tangent}


\begin{multline}
\\
\ln(\arctan(x + x) ^ {2}) ^ {2}.\\
\end{multline}


Lets find the tangent!

We must differentiate expression to find tangent parameters.

Starting differentiation... 
For any listener of Lukashov, at the sight of this expression immediately comes to mind 

\begin{multline}
\\
A_{1} \cdot B_{1} \cdot 2 \cdot (C_{1}).\\
A_{1} = \frac{1}{1 + (x + x) ^ {2}} \cdot 2 \cdot (\arctan(x + x) ^ {2 - 1}).\\
B_{1} = \frac{1}{\arctan(x + x) ^ {2}}.\\
C_{1} = \ln(\arctan(x + x) ^ {2}) ^ {2 - 1}.\\
\end{multline}


Lets simplify this expression.
A fitness trainer from Simferopol\cite{SJ} threatens to beat you if you won't continue the transformation 

\begin{multline}
\\
A_{1} \cdot B_{1} \cdot 2 \cdot C_{1}.\\
A_{1} = \frac{1}{1 + (x + x) ^ {2}} \cdot 2 \cdot \arctan(x + x).\\
B_{1} = \frac{1}{\arctan(x + x) ^ {2}}.\\
C_{1} = \ln(\arctan(x + x) ^ {2}).\\
\end{multline}

This explanation is available only for premium readers of this article (4862 8784 4592 1552) 

\begin{multline}
\\
0.848552 + x \cdot -1.23028.\\
\end{multline}


\includegraphics[scale = 0.5]{img/img1_13712.png}
\section{Getting superhard Taylor series}

Lets find Taylor series of:


\begin{multline}
\\
\sin(x) + \cos(x).\\
\end{multline}


We need to differentiate this:


\begin{multline}
\\
\sin(x) + \cos(x).\\
\end{multline}


Starting differentiation... 
Looks impressive. Still not as impressive as this dance from tiktok\cite{Zolo}, so, we must made another transformation 

\begin{multline}
\\
1 \cdot \cos(x) + -1 \cdot 1 \cdot \sin(x).\\
\end{multline}


Lets simplify this expression.
ARE YOU SURPRISED????\cite{Dashkov} It is clear to the hedgehog that this is the same as 

\begin{multline}
\\
\cos(x) + -1 \cdot \sin(x).\\
\end{multline}


We need to differentiate this:


\begin{multline}
\\
\cos(x) + -1 \cdot \sin(x).\\
\end{multline}


Starting differentiation... 
I would justify this transition, but the article will be more useful if you do it yourself 

\begin{multline}
\\
-1 \cdot 1 \cdot \sin(x) + 0 \cdot \sin(x) + -1 \cdot 1 \cdot \cos(x).\\
\end{multline}


Lets simplify this expression.
A fitness trainer from Simferopol\cite{SJ} threatens to beat you if you won't continue the transformation 

\begin{multline}
\\
-1 \cdot \sin(x) + -1 \cdot \cos(x).\\
\end{multline}


We need to differentiate this:


\begin{multline}
\\
-1 \cdot \sin(x) + -1 \cdot \cos(x).\\
\end{multline}


Starting differentiation... 
At that very lecture you missed, it was proved that this is equal to 

\begin{multline}
\\
A_{1} + B_{1} + C_{1} + D_{1}.\\
A_{1} = 0 \cdot \sin(x).\\
B_{1} = -1 \cdot 1 \cdot \cos(x).\\
C_{1} = 0 \cdot \cos(x).\\
D_{1} = -1 \cdot -1 \cdot 1 \cdot \sin(x).\\
\end{multline}


Lets simplify this expression.
I would justify this transition, but the article will be more useful if you do it yourself 

\begin{multline}
\\
-1 \cdot \cos(x) + -1 \cdot -1 \cdot \sin(x).\\
\end{multline}


We need to differentiate this:


\begin{multline}
\\
-1 \cdot \cos(x) + -1 \cdot -1 \cdot \sin(x).\\
\end{multline}


Starting differentiation... 
I would justify this transition, but the article will be more useful if you do it yourself 

\begin{multline}
\\
A_{1} + B_{1} + C_{1} + D_{1}.\\
A_{1} = 0 \cdot \cos(x).\\
B_{1} = -1 \cdot -1 \cdot 1 \cdot \sin(x).\\
C_{1} = 0 \cdot -1 \cdot \sin(x).\\
D_{1} = -1 \cdot (0 \cdot \sin(x) + -1 \cdot 1 \cdot \cos(x)).\\
\end{multline}


Lets simplify this expression.
At that very lecture you missed, it was proved that this is equal to 

\begin{multline}
\\
-1 \cdot -1 \cdot \sin(x) + -1 \cdot -1 \cdot \cos(x).\\
\end{multline}


We need to differentiate this:


\begin{multline}
\\
-1 \cdot -1 \cdot \sin(x) + -1 \cdot -1 \cdot \cos(x).\\
\end{multline}


Starting differentiation... 
Some guy from asylum \cite{Anton} told me that this is equal to 

\begin{multline}
\\
A_{1} + B_{1} + C_{1} + D_{1}.\\
A_{1} = 0 \cdot -1 \cdot \sin(x).\\
B_{1} = -1 \cdot (0 \cdot \sin(x) + -1 \cdot 1 \cdot \cos(x)).\\
C_{1} = 0 \cdot -1 \cdot \cos(x).\\
D_{1} = -1 \cdot (A_{5} + B_{5}).\\
A_{5} = 0 \cdot \cos(x).\\
B_{5} = -1 \cdot -1 \cdot 1 \cdot \sin(x).\\
\end{multline}


Lets simplify this expression.
ARE YOU SURPRISED????\cite{Dashkov} It is clear to the hedgehog that this is the same as 

\begin{multline}
\\
-1 \cdot -1 \cdot \cos(x) + -1 \cdot -1 \cdot -1 \cdot \sin(x).\\
\end{multline}


We need to differentiate this:


\begin{multline}
\\
-1 \cdot -1 \cdot \cos(x) + -1 \cdot -1 \cdot -1 \cdot \sin(x).\\
\end{multline}


Starting differentiation... 
Are you really still reading this? 

\begin{multline}
\\
A_{1} + B_{1} + C_{1} + D_{1}.\\
A_{1} = 0 \cdot -1 \cdot \cos(x).\\
B_{1} = -1 \cdot (A_{3} + B_{3}).\\
A_{3} = 0 \cdot \cos(x).\\
B_{3} = -1 \cdot -1 \cdot 1 \cdot \sin(x).\\
C_{1} = 0 \cdot -1 \cdot -1 \cdot \sin(x).\\
D_{1} = -1 \cdot (A_{5} + B_{5}).\\
A_{5} = 0 \cdot -1 \cdot \sin(x).\\
B_{5} = -1 \cdot (0 \cdot \sin(x) + -1 \cdot 1 \cdot \cos(x)).\\
\end{multline}


Lets simplify this expression.
ARE YOU SURPRISED????\cite{Dashkov} It is clear to the hedgehog that this is the same as 

\begin{multline}
\\
-1 \cdot -1 \cdot -1 \cdot \sin(x) + -1 \cdot -1 \cdot -1 \cdot \cos(x).\\
\end{multline}


Lets simplify this expression.
ARE YOU SURPRISED????\cite{Dashkov} It is clear to the hedgehog that this is the same as 

\begin{multline}
\\
A_{1} + B_{1} + -0.00942594 \cdot (C_{1}).\\
A_{1} = -0.848872 \cdot 1 + -1.13111 \cdot (x - 3) + 0.424436 \cdot ((x - 3) ^ {2}) + 0.188519 \cdot ((x - 3) ^ {3}).\\
B_{1} = -0.0353697 \cdot ((x - 3) ^ {4}).\\
C_{1} = (x - 3) ^ {5}.\\
\end{multline}

Taylor series is: 
$-0.848872 + -1.13111 \cdot (x - 3) + 0.424436 \cdot ((x - 3) ^ {2}) + 0.188519 \cdot ((x - 3) ^ {3}) + -0.0353697 \cdot ((x - 3) ^ {4}) + -0.00942594 \cdot ((x - 3) ^ {5})+ o((x - 3)^{5}).$\\

\includegraphics[scale = 0.5]{img/img2_14248.png}

Lets find out difference between:


\begin{multline}
\\
\sin(x) + \cos(x).\\
\end{multline}


and


\begin{multline}
\\
A_{1} + B_{1} + -0.00942594 \cdot (C_{1}).\\
A_{1} = -0.848872 + -1.13111 \cdot (x - 3) + 0.424436 \cdot ((x - 3) ^ {2}) + 0.188519 \cdot ((x - 3) ^ {3}).\\
B_{1} = -0.0353697 \cdot ((x - 3) ^ {4}).\\
C_{1} = (x - 3) ^ {5}.\\
\end{multline}

Let's not bother with obvious proof that this is 

\begin{multline}
\\
A_{1} + B_{1} - C_{1} + D_{1}.\\
A_{1} = \sin(x).\\
B_{1} = \cos(x).\\
C_{1} = A_{4} + B_{4} + -0.0353697 \cdot (C_{4}).\\
A_{4} = -0.848872 + -1.13111 \cdot (x - 3) + 0.424436 \cdot ((x - 3) ^ {2}).\\
B_{4} = 0.188519 \cdot ((x - 3) ^ {3}).\\
C_{4} = (x - 3) ^ {4}.\\
D_{1} = -0.00942594 \cdot ((x - 3) ^ {5}).\\
\end{multline}


Lets simplify this expression.

Oopsie, our expression is already too awesome.

\includegraphics[scale = 0.5]{img/img3_14562.png}
\section{Calculating too easy differentiation}


\begin{multline}
\\
\ln(\arctan(x + x) ^ {2}) ^ {2}.\\
\end{multline}


LET'S DIFFERENTIATE THIS!!!
Are you really still reading this? 

\begin{multline}
\\
A_{1} \cdot B_{1} \cdot 2 \cdot (C_{1}).\\
A_{1} = \frac{1}{1 + (x + x) ^ {2}} \cdot 2 \cdot (\arctan(x + x) ^ {2 - 1}).\\
B_{1} = \frac{1}{\arctan(x + x) ^ {2}}.\\
C_{1} = \ln(\arctan(x + x) ^ {2}) ^ {2 - 1}.\\
\end{multline}


Lets simplify this expression.
For any listener of Lukashov, at the sight of this expression immediately comes to mind 

\begin{multline}
\\
A_{1} \cdot B_{1} \cdot 2 \cdot C_{1}.\\
A_{1} = \frac{1}{1 + (x + x) ^ {2}} \cdot 2 \cdot \arctan(x + x).\\
B_{1} = \frac{1}{\arctan(x + x) ^ {2}}.\\
C_{1} = \ln(\arctan(x + x) ^ {2}).\\
\end{multline}


LET'S DIFFERENTIATE THIS!!!
At that very lecture you missed, it was proved that this is equal to 

\begin{multline}
\\
(A_{1}) \cdot B_{1} + C_{1} \cdot (D_{1}).\\
A_{1} = (A_{2}) \cdot B_{2} + C_{2} \cdot D_{2}.\\
A_{2} = A_{3} \cdot B_{3} + C_{3} \cdot (D_{3}).\\
A_{3} = \frac{A_{4} - B_{4}}{(C_{4}) ^ {2}}.\\
A_{4} = 0 \cdot (1 + (x + x) ^ {2}).\\
B_{4} = 1 \cdot (0 + A_{6}).\\
A_{6} = (1 + 1) \cdot 2 \cdot ((x + x) ^ {2 - 1}).\\
C_{4} = 1 + (x + x) ^ {2}.\\
B_{3} = 2 \cdot \arctan(x + x).\\
C_{3} = \frac{1}{1 + (x + x) ^ {2}}.\\
D_{3} = 0 \cdot A_{7} + 2 \cdot B_{7}.\\
A_{7} = \arctan(x + x).\\
B_{7} = \frac{1}{1 + (x + x) ^ {2}}.\\
B_{2} = \frac{1}{\arctan(x + x) ^ {2}}.\\
C_{2} = \frac{1}{1 + (x + x) ^ {2}} \cdot 2 \cdot \arctan(x + x).\\
D_{2} = \frac{A_{6} - B_{6}}{(C_{6}) ^ {2}}.\\
A_{6} = 0 \cdot (\arctan(x + x) ^ {2}).\\
B_{6} = 1 \cdot A_{8} \cdot B_{8}.\\
A_{8} = \frac{1}{1 + (x + x) ^ {2}}.\\
B_{8} = 2 \cdot (\arctan(x + x) ^ {2 - 1}).\\
C_{6} = \arctan(x + x) ^ {2}.\\
B_{1} = 2 \cdot \ln(\arctan(x + x) ^ {2}).\\
C_{1} = A_{4} \cdot B_{4} \cdot \frac{1}{C_{4}}.\\
A_{4} = \frac{1}{1 + (x + x) ^ {2}}.\\
B_{4} = 2 \cdot \arctan(x + x).\\
\end{multline}
\begin{multline}
\\
C_{4} = \arctan(x + x) ^ {2}.\\
D_{1} = 0 \cdot A_{5} + 2 \cdot B_{5}.\\
A_{5} = \ln(\arctan(x + x) ^ {2}).\\
B_{5} = A_{7} \cdot B_{7} \cdot \frac{1}{C_{7}}.\\
A_{7} = \frac{1}{1 + (x + x) ^ {2}}.\\
B_{7} = 2 \cdot (\arctan(x + x) ^ {2 - 1}).\\
C_{7} = \arctan(x + x) ^ {2}.\\
\end{multline}


Lets simplify this expression.
Looks impressive. Still not as impressive as this dance from tiktok\cite{Zolo}, so, we must made another transformation 

\begin{multline}
\\
(A_{1}) \cdot B_{1} + C_{1} \cdot D_{1}.\\
A_{1} = (A_{2}) \cdot B_{2} + C_{2} \cdot D_{2}.\\
A_{2} = A_{3} \cdot B_{3} + C_{3} \cdot D_{3}.\\
A_{3} = \frac{0 - 2 \cdot 2 \cdot (x + x)}{(1 + (x + x) ^ {2}) ^ {2}}.\\
B_{3} = 2 \cdot \arctan(x + x).\\
C_{3} = \frac{1}{1 + (x + x) ^ {2}}.\\
D_{3} = 2 \cdot \frac{1}{1 + (x + x) ^ {2}}.\\
B_{2} = \frac{1}{\arctan(x + x) ^ {2}}.\\
C_{2} = \frac{1}{1 + (x + x) ^ {2}} \cdot 2 \cdot \arctan(x + x).\\
D_{2} = \frac{0 - A_{6}}{(B_{6}) ^ {2}}.\\
A_{6} = \frac{1}{1 + (x + x) ^ {2}} \cdot 2 \cdot \arctan(x + x).\\
B_{6} = \arctan(x + x) ^ {2}.\\
B_{1} = 2 \cdot \ln(\arctan(x + x) ^ {2}).\\
C_{1} = A_{4} \cdot B_{4} \cdot \frac{1}{C_{4}}.\\
A_{4} = \frac{1}{1 + (x + x) ^ {2}}.\\
B_{4} = 2 \cdot \arctan(x + x).\\
C_{4} = \arctan(x + x) ^ {2}.\\
D_{1} = 2 \cdot A_{5} \cdot B_{5}.\\
A_{5} = \frac{1}{1 + (x + x) ^ {2}} \cdot 2 \cdot \arctan(x + x).\\
B_{5} = \frac{1}{\arctan(x + x) ^ {2}}.\\
\end{multline}


LET'S DIFFERENTIATE THIS!!!
Let's not bother with obvious proof that this is 

\begin{multline}
\\
A_{1} + B_{1} + C_{1} + D_{1}.\\
A_{1} = (A_{2} + B_{2}) \cdot 2 \cdot C_{2}.\\
A_{2} = (A_{3}) \cdot B_{3} + (C_{3}) \cdot D_{3}.\\
A_{3} = A_{4} + B_{4} + C_{4} + D_{4}.\\
A_{4} = \frac{A_{5}}{B_{5}} \cdot 2 \cdot C_{5}.\\
A_{5} = (A_{6}) \cdot (B_{6}) - (C_{6}) \cdot D_{6}.\\
A_{6} = 0 - A_{7} + B_{7}.\\
A_{7} = 0 \cdot 2 \cdot (x + x).\\
B_{7} = 2 \cdot (0 \cdot (x + x) + 2 \cdot (1 + 1)).\\
B_{6} = (1 + (x + x) ^ {2}) ^ {2}.\\
C_{6} = 0 - 2 \cdot 2 \cdot (x + x).\\
D_{6} = (0 + A_{10}) \cdot 2 \cdot (B_{10}).\\
A_{10} = (1 + 1) \cdot 2 \cdot ((x + x) ^ {2 - 1}).\\
B_{10} = (1 + (x + x) ^ {2}) ^ {2 - 1}.\\
B_{5} = ((1 + (x + x) ^ {2}) ^ {2}) ^ {2}.\\
C_{5} = \arctan(x + x).\\
B_{4} = \frac{A_{6}}{B_{6}} \cdot (C_{6} + D_{6}).\\
A_{6} = 0 - 2 \cdot 2 \cdot (x + x).\\
B_{6} = (1 + (x + x) ^ {2}) ^ {2}.\\
C_{6} = 0 \cdot \arctan(x + x).\\
D_{6} = 2 \cdot \frac{1}{1 + (x + x) ^ {2}}.\\
C_{4} = \frac{A_{7}}{B_{7}} \cdot 2 \cdot C_{7}.\\
A_{7} = 0 \cdot (A_{8}) - 1 \cdot (B_{8}).\\
A_{8} = 1 + (x + x) ^ {2}.\\
B_{8} = 0 + (1 + 1) \cdot 2 \cdot ((x + x) ^ {2 - 1}).\\
\end{multline}
\begin{multline}
\\
B_{7} = (1 + (x + x) ^ {2}) ^ {2}.\\
C_{7} = \frac{1}{1 + (x + x) ^ {2}}.\\
D_{4} = \frac{1}{A_{8}} \cdot (B_{8} + C_{8}).\\
A_{8} = 1 + (x + x) ^ {2}.\\
B_{8} = 0 \cdot \frac{1}{1 + (x + x) ^ {2}}.\\
C_{8} = 2 \cdot \frac{A_{11}}{B_{11}}.\\
A_{11} = 0 \cdot (A_{12}) - 1 \cdot (B_{12}).\\
A_{12} = 1 + (x + x) ^ {2}.\\
B_{12} = 0 + (1 + 1) \cdot 2 \cdot ((x + x) ^ {2 - 1}).\\
B_{11} = (1 + (x + x) ^ {2}) ^ {2}.\\
B_{3} = \frac{1}{\arctan(x + x) ^ {2}}.\\
C_{3} = A_{6} \cdot B_{6} + C_{6} \cdot D_{6}.\\
A_{6} = \frac{0 - 2 \cdot 2 \cdot (x + x)}{(1 + (x + x) ^ {2}) ^ {2}}.\\
B_{6} = 2 \cdot \arctan(x + x).\\
C_{6} = \frac{1}{1 + (x + x) ^ {2}}.\\
D_{6} = 2 \cdot \frac{1}{1 + (x + x) ^ {2}}.\\
D_{3} = \frac{A_{7} - B_{7}}{(C_{7}) ^ {2}}.\\
A_{7} = 0 \cdot (\arctan(x + x) ^ {2}).\\
B_{7} = 1 \cdot A_{9} \cdot B_{9}.\\
A_{9} = \frac{1}{1 + (x + x) ^ {2}}.\\
B_{9} = 2 \cdot (\arctan(x + x) ^ {2 - 1}).\\
C_{7} = \arctan(x + x) ^ {2}.\\
B_{2} = (A_{4}) \cdot B_{4} + C_{4} \cdot D_{4}.\\
A_{4} = A_{5} \cdot B_{5} + C_{5} \cdot (D_{5}).\\
A_{5} = \frac{A_{6} - B_{6}}{(C_{6}) ^ {2}}.\\
\end{multline}
\begin{multline}
\\
A_{6} = 0 \cdot (1 + (x + x) ^ {2}).\\
B_{6} = 1 \cdot (0 + A_{8}).\\
A_{8} = (1 + 1) \cdot 2 \cdot ((x + x) ^ {2 - 1}).\\
C_{6} = 1 + (x + x) ^ {2}.\\
B_{5} = 2 \cdot \arctan(x + x).\\
C_{5} = \frac{1}{1 + (x + x) ^ {2}}.\\
D_{5} = 0 \cdot A_{9} + 2 \cdot B_{9}.\\
A_{9} = \arctan(x + x).\\
B_{9} = \frac{1}{1 + (x + x) ^ {2}}.\\
B_{4} = \frac{0 - A_{6}}{(B_{6}) ^ {2}}.\\
A_{6} = \frac{1}{1 + (x + x) ^ {2}} \cdot 2 \cdot \arctan(x + x).\\
B_{6} = \arctan(x + x) ^ {2}.\\
C_{4} = \frac{1}{1 + (x + x) ^ {2}} \cdot 2 \cdot \arctan(x + x).\\
D_{4} = \frac{A_{8} - B_{8}}{(C_{8}) ^ {2}}.\\
A_{8} = (0 - A_{9}) \cdot ((B_{9}) ^ {2}).\\
A_{9} = A_{10} \cdot B_{10} + C_{10} \cdot (D_{10}).\\
A_{10} = \frac{A_{11} - B_{11}}{(C_{11}) ^ {2}}.\\
A_{11} = 0 \cdot (1 + (x + x) ^ {2}).\\
B_{11} = 1 \cdot (0 + A_{13}).\\
A_{13} = (1 + 1) \cdot 2 \cdot ((x + x) ^ {2 - 1}).\\
C_{11} = 1 + (x + x) ^ {2}.\\
B_{10} = 2 \cdot \arctan(x + x).\\
C_{10} = \frac{1}{1 + (x + x) ^ {2}}.\\
D_{10} = 0 \cdot A_{14} + 2 \cdot B_{14}.\\
A_{14} = \arctan(x + x).\\
\end{multline}
\begin{multline}
\\
B_{14} = \frac{1}{1 + (x + x) ^ {2}}.\\
B_{9} = \arctan(x + x) ^ {2}.\\
B_{8} = (0 - A_{10}) \cdot B_{10} \cdot C_{10}.\\
A_{10} = \frac{1}{1 + (x + x) ^ {2}} \cdot 2 \cdot \arctan(x + x).\\
B_{10} = \frac{1}{1 + (x + x) ^ {2}} \cdot 2 \cdot (\arctan(x + x) ^ {2 - 1}).\\
C_{10} = 2 \cdot ((\arctan(x + x) ^ {2}) ^ {2 - 1}).\\
C_{8} = (\arctan(x + x) ^ {2}) ^ {2}.\\
C_{2} = \ln(\arctan(x + x) ^ {2}).\\
B_{1} = (A_{3} + B_{3}) \cdot (C_{3} + D_{3}).\\
A_{3} = (A_{4} + B_{4}) \cdot \frac{1}{C_{4}}.\\
A_{4} = \frac{A_{5}}{B_{5}} \cdot 2 \cdot C_{5}.\\
A_{5} = 0 - 2 \cdot 2 \cdot (x + x).\\
B_{5} = (1 + (x + x) ^ {2}) ^ {2}.\\
C_{5} = \arctan(x + x).\\
B_{4} = \frac{1}{A_{6}} \cdot 2 \cdot B_{6}.\\
A_{6} = 1 + (x + x) ^ {2}.\\
B_{6} = \frac{1}{1 + (x + x) ^ {2}}.\\
C_{4} = \arctan(x + x) ^ {2}.\\
B_{3} = A_{5} \cdot B_{5} \cdot \frac{C_{5}}{D_{5}}.\\
A_{5} = \frac{1}{1 + (x + x) ^ {2}}.\\
B_{5} = 2 \cdot \arctan(x + x).\\
C_{5} = 0 - A_{8} \cdot B_{8}.\\
A_{8} = \frac{1}{1 + (x + x) ^ {2}}.\\
B_{8} = 2 \cdot \arctan(x + x).\\
D_{5} = (\arctan(x + x) ^ {2}) ^ {2}.\\
\end{multline}
\begin{multline}
\\
C_{3} = 0 \cdot \ln(\arctan(x + x) ^ {2}).\\
D_{3} = 2 \cdot A_{7} \cdot B_{7}.\\
A_{7} = \frac{1}{1 + (x + x) ^ {2}} \cdot 2 \cdot (\arctan(x + x) ^ {2 - 1}).\\
B_{7} = \frac{1}{\arctan(x + x) ^ {2}}.\\
C_{1} = (A_{4} + B_{4}) \cdot 2 \cdot C_{4}.\\
A_{4} = (A_{5} + B_{5}) \cdot \frac{1}{C_{5}}.\\
A_{5} = \frac{A_{6}}{B_{6}} \cdot 2 \cdot C_{6}.\\
A_{6} = 0 \cdot (A_{7}) - 1 \cdot (B_{7}).\\
A_{7} = 1 + (x + x) ^ {2}.\\
B_{7} = 0 + (1 + 1) \cdot 2 \cdot ((x + x) ^ {2 - 1}).\\
B_{6} = (1 + (x + x) ^ {2}) ^ {2}.\\
C_{6} = \arctan(x + x).\\
B_{5} = \frac{1}{A_{7}} \cdot (B_{7} + C_{7}).\\
A_{7} = 1 + (x + x) ^ {2}.\\
B_{7} = 0 \cdot \arctan(x + x).\\
C_{7} = 2 \cdot \frac{1}{1 + (x + x) ^ {2}}.\\
C_{5} = \arctan(x + x) ^ {2}.\\
B_{4} = A_{6} \cdot B_{6} \cdot \frac{C_{6}}{D_{6}}.\\
A_{6} = \frac{1}{1 + (x + x) ^ {2}}.\\
B_{6} = 2 \cdot \arctan(x + x).\\
C_{6} = 0 \cdot (A_{9}) - 1 \cdot B_{9}.\\
A_{9} = \arctan(x + x) ^ {2}.\\
B_{9} = \frac{1}{1 + (x + x) ^ {2}} \cdot 2 \cdot (\arctan(x + x) ^ {2 - 1}).\\
D_{6} = (\arctan(x + x) ^ {2}) ^ {2}.\\
C_{4} = A_{7} \cdot B_{7} \cdot \frac{1}{C_{7}}.\\
\end{multline}
\begin{multline}
\\
A_{7} = \frac{1}{1 + (x + x) ^ {2}}.\\
B_{7} = 2 \cdot \arctan(x + x).\\
C_{7} = \arctan(x + x) ^ {2}.\\
D_{1} = A_{5} \cdot B_{5} \cdot (C_{5} + D_{5}).\\
A_{5} = \frac{1}{1 + (x + x) ^ {2}} \cdot 2 \cdot \arctan(x + x).\\
B_{5} = \frac{1}{\arctan(x + x) ^ {2}}.\\
C_{5} = 0 \cdot A_{8} \cdot B_{8}.\\
A_{8} = \frac{1}{1 + (x + x) ^ {2}} \cdot 2 \cdot \arctan(x + x).\\
B_{8} = \frac{1}{\arctan(x + x) ^ {2}}.\\
D_{5} = 2 \cdot (A_{9} + B_{9}).\\
A_{9} = (A_{10} + B_{10}) \cdot \frac{1}{C_{10}}.\\
A_{10} = \frac{A_{11}}{B_{11}} \cdot 2 \cdot C_{11}.\\
A_{11} = 0 \cdot (A_{12}) - 1 \cdot (B_{12}).\\
A_{12} = 1 + (x + x) ^ {2}.\\
B_{12} = 0 + (1 + 1) \cdot 2 \cdot ((x + x) ^ {2 - 1}).\\
B_{11} = (1 + (x + x) ^ {2}) ^ {2}.\\
C_{11} = \arctan(x + x).\\
B_{10} = \frac{1}{A_{12}} \cdot (B_{12} + C_{12}).\\
A_{12} = 1 + (x + x) ^ {2}.\\
B_{12} = 0 \cdot \arctan(x + x).\\
C_{12} = 2 \cdot \frac{1}{1 + (x + x) ^ {2}}.\\
C_{10} = \arctan(x + x) ^ {2}.\\
B_{9} = A_{11} \cdot B_{11} \cdot \frac{C_{11}}{D_{11}}.\\
A_{11} = \frac{1}{1 + (x + x) ^ {2}}.\\
B_{11} = 2 \cdot \arctan(x + x).\\
\end{multline}
\begin{multline}
\\
C_{11} = 0 \cdot (A_{14}) - 1 \cdot B_{14}.\\
A_{14} = \arctan(x + x) ^ {2}.\\
B_{14} = \frac{1}{1 + (x + x) ^ {2}} \cdot 2 \cdot (\arctan(x + x) ^ {2 - 1}).\\
D_{11} = (\arctan(x + x) ^ {2}) ^ {2}.\\
\end{multline}


Lets simplify this expression.
This explanation is available only for premium readers of this article (4862 8784 4592 1552) 

\begin{multline}
\\
A_{1} + B_{1} + C_{1} + D_{1}.\\
A_{1} = (A_{2} + B_{2}) \cdot 2 \cdot C_{2}.\\
A_{2} = (A_{3}) \cdot B_{3} + (C_{3}) \cdot D_{3}.\\
A_{3} = A_{4} + B_{4} + C_{4} + D_{4}.\\
A_{4} = \frac{A_{5}}{B_{5}} \cdot 2 \cdot C_{5}.\\
A_{5} = (A_{6}) \cdot (B_{6}) - (C_{6}) \cdot D_{6}.\\
A_{6} = 0 - 2 \cdot 4.\\
B_{6} = (1 + (x + x) ^ {2}) ^ {2}.\\
C_{6} = 0 - 2 \cdot 2 \cdot (x + x).\\
D_{6} = 2 \cdot 2 \cdot (x + x) \cdot 2 \cdot (1 + (x + x) ^ {2}).\\
B_{5} = ((1 + (x + x) ^ {2}) ^ {2}) ^ {2}.\\
C_{5} = \arctan(x + x).\\
B_{4} = \frac{A_{6}}{B_{6}} \cdot 2 \cdot C_{6}.\\
A_{6} = 0 - 2 \cdot 2 \cdot (x + x).\\
B_{6} = (1 + (x + x) ^ {2}) ^ {2}.\\
C_{6} = \frac{1}{1 + (x + x) ^ {2}}.\\
C_{4} = \frac{A_{7}}{B_{7}} \cdot 2 \cdot C_{7}.\\
A_{7} = 0 - 2 \cdot 2 \cdot (x + x).\\
B_{7} = (1 + (x + x) ^ {2}) ^ {2}.\\
C_{7} = \frac{1}{1 + (x + x) ^ {2}}.\\
D_{4} = \frac{1}{A_{8}} \cdot 2 \cdot B_{8}.\\
A_{8} = 1 + (x + x) ^ {2}.\\
B_{8} = \frac{0 - 2 \cdot 2 \cdot (x + x)}{(1 + (x + x) ^ {2}) ^ {2}}.\\
B_{3} = \frac{1}{\arctan(x + x) ^ {2}}.\\
C_{3} = A_{6} \cdot B_{6} + C_{6} \cdot D_{6}.\\
\end{multline}
\begin{multline}
\\
A_{6} = \frac{0 - 2 \cdot 2 \cdot (x + x)}{(1 + (x + x) ^ {2}) ^ {2}}.\\
B_{6} = 2 \cdot \arctan(x + x).\\
C_{6} = \frac{1}{1 + (x + x) ^ {2}}.\\
D_{6} = 2 \cdot \frac{1}{1 + (x + x) ^ {2}}.\\
D_{3} = \frac{0 - A_{7}}{(B_{7}) ^ {2}}.\\
A_{7} = \frac{1}{1 + (x + x) ^ {2}} \cdot 2 \cdot \arctan(x + x).\\
B_{7} = \arctan(x + x) ^ {2}.\\
B_{2} = (A_{4}) \cdot B_{4} + C_{4} \cdot D_{4}.\\
A_{4} = A_{5} \cdot B_{5} + C_{5} \cdot D_{5}.\\
A_{5} = \frac{0 - 2 \cdot 2 \cdot (x + x)}{(1 + (x + x) ^ {2}) ^ {2}}.\\
B_{5} = 2 \cdot \arctan(x + x).\\
C_{5} = \frac{1}{1 + (x + x) ^ {2}}.\\
D_{5} = 2 \cdot \frac{1}{1 + (x + x) ^ {2}}.\\
B_{4} = \frac{0 - A_{6}}{(B_{6}) ^ {2}}.\\
A_{6} = \frac{1}{1 + (x + x) ^ {2}} \cdot 2 \cdot \arctan(x + x).\\
B_{6} = \arctan(x + x) ^ {2}.\\
C_{4} = \frac{1}{1 + (x + x) ^ {2}} \cdot 2 \cdot \arctan(x + x).\\
D_{4} = \frac{A_{8} - B_{8}}{(C_{8}) ^ {2}}.\\
A_{8} = (0 - A_{9}) \cdot ((B_{9}) ^ {2}).\\
A_{9} = A_{10} \cdot B_{10} + C_{10} \cdot D_{10}.\\
A_{10} = \frac{0 - 2 \cdot 2 \cdot (x + x)}{(1 + (x + x) ^ {2}) ^ {2}}.\\
B_{10} = 2 \cdot \arctan(x + x).\\
C_{10} = \frac{1}{1 + (x + x) ^ {2}}.\\
D_{10} = 2 \cdot \frac{1}{1 + (x + x) ^ {2}}.\\
B_{9} = \arctan(x + x) ^ {2}.\\
\end{multline}
\begin{multline}
\\
B_{8} = (0 - A_{10}) \cdot B_{10} \cdot C_{10}.\\
A_{10} = \frac{1}{1 + (x + x) ^ {2}} \cdot 2 \cdot \arctan(x + x).\\
B_{10} = \frac{1}{1 + (x + x) ^ {2}} \cdot 2 \cdot \arctan(x + x).\\
C_{10} = 2 \cdot (\arctan(x + x) ^ {2}).\\
C_{8} = (\arctan(x + x) ^ {2}) ^ {2}.\\
C_{2} = \ln(\arctan(x + x) ^ {2}).\\
B_{1} = (A_{3} + B_{3}) \cdot 2 \cdot C_{3}.\\
A_{3} = (A_{4} + B_{4}) \cdot \frac{1}{C_{4}}.\\
A_{4} = \frac{A_{5}}{B_{5}} \cdot 2 \cdot C_{5}.\\
A_{5} = 0 - 2 \cdot 2 \cdot (x + x).\\
B_{5} = (1 + (x + x) ^ {2}) ^ {2}.\\
C_{5} = \arctan(x + x).\\
B_{4} = \frac{1}{A_{6}} \cdot 2 \cdot B_{6}.\\
A_{6} = 1 + (x + x) ^ {2}.\\
B_{6} = \frac{1}{1 + (x + x) ^ {2}}.\\
C_{4} = \arctan(x + x) ^ {2}.\\
B_{3} = A_{5} \cdot B_{5} \cdot \frac{C_{5}}{D_{5}}.\\
A_{5} = \frac{1}{1 + (x + x) ^ {2}}.\\
B_{5} = 2 \cdot \arctan(x + x).\\
C_{5} = 0 - A_{8} \cdot B_{8}.\\
A_{8} = \frac{1}{1 + (x + x) ^ {2}}.\\
B_{8} = 2 \cdot \arctan(x + x).\\
D_{5} = (\arctan(x + x) ^ {2}) ^ {2}.\\
C_{3} = A_{6} \cdot B_{6} \cdot \frac{1}{C_{6}}.\\
A_{6} = \frac{1}{1 + (x + x) ^ {2}}.\\
\end{multline}
\begin{multline}
\\
B_{6} = 2 \cdot \arctan(x + x).\\
C_{6} = \arctan(x + x) ^ {2}.\\
C_{1} = (A_{4} + B_{4}) \cdot 2 \cdot C_{4}.\\
A_{4} = (A_{5} + B_{5}) \cdot \frac{1}{C_{5}}.\\
A_{5} = \frac{A_{6}}{B_{6}} \cdot 2 \cdot C_{6}.\\
A_{6} = 0 - 2 \cdot 2 \cdot (x + x).\\
B_{6} = (1 + (x + x) ^ {2}) ^ {2}.\\
C_{6} = \arctan(x + x).\\
B_{5} = \frac{1}{A_{7}} \cdot 2 \cdot B_{7}.\\
A_{7} = 1 + (x + x) ^ {2}.\\
B_{7} = \frac{1}{1 + (x + x) ^ {2}}.\\
C_{5} = \arctan(x + x) ^ {2}.\\
B_{4} = A_{6} \cdot B_{6} \cdot \frac{C_{6}}{D_{6}}.\\
A_{6} = \frac{1}{1 + (x + x) ^ {2}}.\\
B_{6} = 2 \cdot \arctan(x + x).\\
C_{6} = 0 - A_{9} \cdot B_{9}.\\
A_{9} = \frac{1}{1 + (x + x) ^ {2}}.\\
B_{9} = 2 \cdot \arctan(x + x).\\
D_{6} = (\arctan(x + x) ^ {2}) ^ {2}.\\
C_{4} = A_{7} \cdot B_{7} \cdot \frac{1}{C_{7}}.\\
A_{7} = \frac{1}{1 + (x + x) ^ {2}}.\\
B_{7} = 2 \cdot \arctan(x + x).\\
C_{7} = \arctan(x + x) ^ {2}.\\
D_{1} = A_{5} \cdot B_{5} \cdot 2 \cdot (C_{5}).\\
A_{5} = \frac{1}{1 + (x + x) ^ {2}} \cdot 2 \cdot \arctan(x + x).\\
\end{multline}
\begin{multline}
\\
B_{5} = \frac{1}{\arctan(x + x) ^ {2}}.\\
C_{5} = (A_{8}) \cdot B_{8} + C_{8} \cdot D_{8}.\\
A_{8} = A_{9} \cdot B_{9} + C_{9} \cdot D_{9}.\\
A_{9} = \frac{0 - 2 \cdot 2 \cdot (x + x)}{(1 + (x + x) ^ {2}) ^ {2}}.\\
B_{9} = 2 \cdot \arctan(x + x).\\
C_{9} = \frac{1}{1 + (x + x) ^ {2}}.\\
D_{9} = 2 \cdot \frac{1}{1 + (x + x) ^ {2}}.\\
B_{8} = \frac{1}{\arctan(x + x) ^ {2}}.\\
C_{8} = \frac{1}{1 + (x + x) ^ {2}} \cdot 2 \cdot \arctan(x + x).\\
D_{8} = \frac{0 - A_{12}}{(B_{12}) ^ {2}}.\\
A_{12} = \frac{1}{1 + (x + x) ^ {2}} \cdot 2 \cdot \arctan(x + x).\\
B_{12} = \arctan(x + x) ^ {2}.\\
\end{multline}

\printbibliography
\end{document}
