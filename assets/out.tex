\documentclass[12pt,a4paper]{extreport}
\input{style}
\title{<<I fucking love science>>}
\author{Maklakov Artyom B05-332}
\usepackage[ sorting = none, style = science ]{biblatex}
\addbibresource{library.bib}
\begin{document}
\maketitle
\tableofcontents
\section{Getting superhard tangent}


\begin{multline}
\ln(\arctan(x + x) ^ {2}) ^ {2}.\\
\end{multline}


Lets find the tangent!

We must differentiate expression to find tangent parameters.

Starting differentiation... 
Let's not bother with obvious proof that this is 

\begin{multline}
\frac{1}{1 + A_{1}} \cdot 2 \cdot (B_{1} ^ {C_{1}}) \cdot \frac{1}{\arctan(D_{1}) ^ {2}} \cdot 2 \cdot (\ln(E_{1} ^ {2}) ^ {2 - 1}).\\
A_{1} = (x + x) ^ {2}.\\
B_{1} = \arctan(x + x).\\
C_{1} = 2 - 1.\\
D_{1} = x + x.\\
E_{1} = \arctan(x + x).\\
\end{multline}


Lets simplify this expression.
A fitness trainer from Simferopol\cite{SJ} threatens to beat you if you don't continue the transformation 

\begin{multline}
\frac{1}{1 + A_{1}} \cdot 2 \cdot \arctan(B_{1}) \cdot \frac{1}{\arctan(C_{1}) ^ {2}} \cdot 2 \cdot \ln(\arctan(D_{1}) ^ {2}).\\
A_{1} = (x + x) ^ {2}.\\
B_{1} = x + x.\\
C_{1} = x + x.\\
D_{1} = x + x.\\
\end{multline}

A fitness trainer from Simferopol\cite{SJ} threatens to beat you if you don't continue the transformation 

\begin{multline}
0.848552 + x \cdot -1.23028.\\
\end{multline}


\includegraphics[scale = 0.5]{img/img1_14071.png}
\section{Getting superhard Taylor series}

Lets find taylor series of:


\begin{multline}
\sin(x) + \cos(x).\\
\end{multline}


We need to differentiate this:


\begin{multline}
\sin(x) + \cos(x).\\
\end{multline}


Starting differentiation... 
A fitness trainer from Simferopol\cite{SJ} threatens to beat you if you don't continue the transformation 

\begin{multline}
1 \cdot \cos(x) + -1 \cdot 1 \cdot \sin(x).\\
\end{multline}


Lets simplify this expression.
Zhirinovsky suggested \cite{Zhirinovsky} to do this simplification 

\begin{multline}
\cos(x) + -1 \cdot \sin(x).\\
\end{multline}


We need to differentiate this:


\begin{multline}
\cos(x) + -1 \cdot \sin(x).\\
\end{multline}


Starting differentiation... 
A fitness trainer from Simferopol\cite{SJ} threatens to beat you if you don't continue the transformation 

\begin{multline}
-1 \cdot 1 \cdot \sin(x) + 0 \cdot \sin(x) + -1 \cdot 1 \cdot \cos(x).\\
\end{multline}


Lets simplify this expression.
This explanation is available only for premium readers of this article (4862 8784 4592 1552) 

\begin{multline}
-1 \cdot \sin(x) + -1 \cdot \cos(x).\\
\end{multline}


We need to differentiate this:


\begin{multline}
-1 \cdot \sin(x) + -1 \cdot \cos(x).\\
\end{multline}


Starting differentiation... 
A fitness trainer from Simferopol\cite{SJ} threatens to beat you if you don't continue the transformation 

\begin{multline}
0 \cdot \sin(x) + -1 \cdot 1 \cdot \cos(x) + 0 \cdot \cos(x) + -1 \cdot -1 \cdot 1 \cdot A_{1}.\\
A_{1} = \sin(x).\\
\end{multline}


Lets simplify this expression.
A fitness trainer from Simferopol\cite{SJ} threatens to beat you if you don't continue the transformation 

\begin{multline}
-1 \cdot \cos(x) + -1 \cdot -1 \cdot \sin(x).\\
\end{multline}


We need to differentiate this:


\begin{multline}
-1 \cdot \cos(x) + -1 \cdot -1 \cdot \sin(x).\\
\end{multline}


Starting differentiation... 
This explanation is available only for premium readers of this article (4862 8784 4592 1552) 

\begin{multline}
0 \cdot \cos(x) + -1 \cdot -1 \cdot 1 \cdot A_{1} + 0 \cdot -1 \cdot \sin(x) + -1 \cdot (0 \cdot B_{1} + -1 \cdot C_{1}).\\
A_{1} = \sin(x).\\
B_{1} = \sin(x).\\
C_{1} = 1 \cdot \cos(x).\\
\end{multline}


Lets simplify this expression.
Some guy from asylum \cite{Anton} told me that this is equal to 

\begin{multline}
-1 \cdot -1 \cdot \sin(x) + -1 \cdot -1 \cdot \cos(x).\\
\end{multline}


We need to differentiate this:


\begin{multline}
-1 \cdot -1 \cdot \sin(x) + -1 \cdot -1 \cdot \cos(x).\\
\end{multline}


Starting differentiation... 
After elementary simplifications, it is obvious that it is equal to 

\begin{multline}
0 \cdot -1 \cdot \sin(x) + -1 \cdot (0 \cdot A_{1} + -1 \cdot B_{1}) + 0 \cdot -1 \cdot \cos(x) + -1 \cdot (0 \cdot C_{1} + -1 \cdot D_{1}).\\
A_{1} = \sin(x).\\
B_{1} = 1 \cdot \cos(x).\\
C_{1} = \cos(x).\\
D_{1} = -1 \cdot 1 \cdot \sin(x).\\
\end{multline}


Lets simplify this expression.
Looks impressive. Still not as impressive as this dance from tiktok\cite{Zolo}, so, we must made another transformation 

\begin{multline}
-1 \cdot -1 \cdot \cos(x) + -1 \cdot -1 \cdot -1 \cdot \sin(x).\\
\end{multline}


We need to differentiate this:


\begin{multline}
-1 \cdot -1 \cdot \cos(x) + -1 \cdot -1 \cdot -1 \cdot \sin(x).\\
\end{multline}


Starting differentiation... 
I would justify this transition, but the article will be more useful if you do it yourself 

\begin{multline}
0 \cdot -1 \cdot \cos(x) + -1 \cdot (0 \cdot A_{1} + -1 \cdot B_{1}) + 0 \cdot -1 \cdot -1 \cdot C_{1} + -1 \cdot (0 \cdot D_{1} + -1 \cdot (E_{1})).\\
A_{1} = \cos(x).\\
B_{1} = -1 \cdot 1 \cdot \sin(x).\\
C_{1} = \sin(x).\\
D_{1} = -1 \cdot \sin(x).\\
E_{1} = 0 \cdot \sin(x) + -1 \cdot 1 \cdot \cos(x).\\
\end{multline}


Lets simplify this expression.
After elementary simplifications, it is obvious that it is equal to 

\begin{multline}
-1 \cdot -1 \cdot -1 \cdot \sin(x) + -1 \cdot -1 \cdot -1 \cdot \cos(x).\\
\end{multline}


Lets simplify this expression.
A fitness trainer from Simferopol\cite{SJ} threatens to beat you if you don't continue the transformation 

\begin{multline}
A_{1} + B_{1} + 0.424436 \cdot (C_{1}) + 0.188519 \cdot ((D_{1}) ^ {3}) + -0.0353697 \cdot ((x - 3) ^ {4}) + -0.00942594 \cdot ((x - 3) ^ {5}).\\
A_{1} = -0.848872 \cdot 1.\\
B_{1} = -1.13111 \cdot (x - 3).\\
C_{1} = (x - 3) ^ {2}.\\
D_{1} = x - 3.\\
\end{multline}

Taylor series is: 
$-0.848872 + -1.13111 \cdot (x - 3) + 0.424436 \cdot ((x - 3) ^ {2}) + 0.188519 \cdot ((x - 3) ^ {3}) + -0.0353697 \cdot ((x - 3) ^ {4}) + -0.00942594 \cdot ((x - 3) ^ {5})+ o((x - 3)^{5}).$\\

\includegraphics[scale = 0.5]{img/img2_14585.png}

Lets find out difference between:


\begin{multline}
\sin(x) + \cos(x).\\
\end{multline}


and


\begin{multline}
-0.848872 + A_{1} + 0.424436 \cdot (B_{1}) + 0.188519 \cdot ((C_{1}) ^ {3}) + -0.0353697 \cdot ((x - 3) ^ {4}) + -0.00942594 \cdot ((x - 3) ^ {5}).\\
A_{1} = -1.13111 \cdot (x - 3).\\
B_{1} = (x - 3) ^ {2}.\\
C_{1} = x - 3.\\
\end{multline}

Zhirinovsky suggested \cite{Zhirinovsky} to do this simplification 

\begin{multline}
\sin(x) + \cos(x) - A_{1} + B_{1} + 0.188519 \cdot (C_{1}) + -0.0353697 \cdot ((D_{1}) ^ {4}) + -0.00942594 \cdot ((x - 3) ^ {5}).\\
A_{1} = -0.848872 + -1.13111 \cdot (x - 3).\\
B_{1} = 0.424436 \cdot ((x - 3) ^ {2}).\\
C_{1} = (x - 3) ^ {3}.\\
D_{1} = x - 3.\\
\end{multline}


Lets simplify this expression.

Oopsie, our expression is already too awesome.

\includegraphics[scale = 0.5]{img/img3_14961.png}
\printbibliography
\end{document}
